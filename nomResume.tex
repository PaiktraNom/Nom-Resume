\documentclass[11pt, a4paper]{resume}
\usepackage{xcolor}
\usepackage{parskip}
\usepackage{hyperref}
\usepackage[a4paper, total={6in, 8in}, margin=0.5in]{geometry}

\name{Paiktra Nom} % Your name
\address{(651)983-3133 \\ \href{paiktranom@gmail.com}{paiktranom@gmail.com} \\ \href{https://paiktranom.github.io}{paiktranom.github.io}}
%=========Career Objective========

\begin{document}
\begin{rSection}{Career Objective}
{\normalfont Interested in a role where I can expand my technical skills to be a well-rounded and experienced software engineer. I am a software engineering professional with front-end experience in developing, deploying, and maintaining web and desktop applications. Passionate about developing innovative and efficient software solutions that meet customer needs.}
\end{rSection}

%========Work Experience=======

\begin{rSection}{Work Experience}
\begin{rSubsection}{Land Processing Distributed Active Archive Center (LP DAAC), USGS}{\normalfont September 2020 - June 2023}{Software Engineer I}{}
 \item {\normalfont Designed solutions to help maintain 10+ applications for the ASTER mission.}
  \item {\normalfont Conducted the operation of updating 100+ outdated libraries for over 12 different applications using several different programming languages.}
 \item {\normalfont Applied Sprint methodology to help with organization of active tasks while transitioning to sAFE methodology.}
 \item {\normalfont Identified and resolved vulnerabilities in ASTER tools without disrupting productivity.}
 \item {\normalfont Coordinated with NASA scientists in requirements gathering for the creation of 150+ JIRA tickets.}
\end{rSubsection}
\begin{rSubsection}{EROS CalVal Center of Excellence, USGS}{\normalfont June 2020 - September 2020}{Software Engineer Intern.}{}
 \item {\normalfont Research for georeferencing design using Landsat 8 data to attach a geographical coordinate system to an aerial image.}
 \item {\normalfont Designed and built an algorithm with ArcGIS for automated production of georeferencing images.}
  \item {\normalfont Collaborated with another member of the ECCOE team to create weekly presentations.}
\end{rSubsection}

\end{rSection}

%============Projects==========
\begin{rSection}{Projects}
\begin{rSubsection}{LP DAAC External Website}{\normalfont September 2020 - June 2023}{Software Engineer}{\href{https://lpdaac.usgs.gov}{https://lpdaac.usgs.gov}}{}
  \item{\normalfont Actively engaged in creating new features and maintaining the website for over 200,000 monthly visitors.}
  \item{\normalfont Collaborated in a 4 person team to improve the user experience for LPDAAC scientists to create news articles, ASTER Products, and E-Learning pages.}
  \item{\normalfont Overhauled the publications table to improve loading times by 23\% as it parses over 10,000 publications.}
  \item{\normalfont Spearheaded the creation of the Podcasts page. Allowing visitors another resource for obtaining information on different NASA missions such as ECOSTRESS and EMIT.}
\end{rSubsection}

\end{rSection}

%===========Education==========

\begin{rSection}{Education}
{\bf South Dakota State University, Brookings, SD} \hfill {\normalfont August 2017 - May 2021} 
\\ {\normalfont B.E., Computer Science with Minors in Mathematics and Software Engineering}\hfill 
\end{rSection}

%========Technical Abilities=======

\begin{rSection}{Technical Abilities}
\begin{tabular}{ @{} >{\bfseries}l @{\hspace{6ex}} l }
Languages \ & {\normalfont C, C++, C\#, Python, HTML5, CSS, Javascript, PHP, Perl}  \\
Frameworks &  {\normalfont Django, Flask, ASP.Net MVC, Wagtail CMS}\\
Tools & {\normalfont Docker, Kubernetes, Jira, Acunetix } \\
Databases & {\normalfont PostgresSQL}\\
Cloud Technologies & {\normalfont AWS}\\
Version Control & {\normalfont Github, GitLab}\\
\\
Soft Skills & {\normalfont Communication, Patience, Teamwork, Agile}
\end{tabular}
\end{rSection}

%========Technical Abilities=======
\begin{rSection}{Awards}
\begin{rSubsection}{}{}{}{}
    \item {Graduated university with Magna Cum Laude distinction: \normalfont2021} \\
    \item {Deans list for 5 semesters.}
\end{rSubsection}

\end{rSection}


\end{document}


